\documentclass[doc,norsk]{apa7}
\usepackage[style=apa]{biblatex}
\usepackage[utf8]{inputenc}
\usepackage[norsk]{babel}


\DeclareLanguageMapping{norsk}{norsk-apa}

\bibliography{references.bib}

\title{Metodologi for replikasjonsprosjekt}
\author{Ole Fredrik Borgundvåg Berg}
\affiliation{NTNU}

\begin{document}

\maketitle

Dette er et replikasjonsprosjekt som er basert på et tilsvarende prosjekt som ble gjort av Ulrich Schimmack, der man prøvde å finne i hvilken grad vitenskaplige funn ved ulike amerikanske universiteter kunne replikeres, inkludert estimert replikasjonsrate på enkeltforskere \parencite{amerikansk-ranking}. Siden faktisk replikasjon er dyrt, så laget heller han med flere en modell som finner estimert replikasjonrate basert på p-verdier i artiklene til forskerene, som er implementert i biblioteket \guillemotleft z-curve\guillemetright\ til R \parencite{z-curve-modell, z-curve-implementasjon}.

\subsection{Finne forskere og artikler}
For å finne artikler og forskere ble Cristin APIet brukt \parencite{cristin-api}. APIet gjør det enkelt både å finne ut hvilke forskere som er tilknyttet hvilke institutt og hvilke artikler som er skrevet av hvilke forskere. Artiklene som er brukt i dette prosjektet er de publikasjonene som har \texttt{category.code = "ARTICLE"} i APIet. Selve nedlastingen av artiklene skjedde også ved programmerisk, men ikke ved hjelp av Cristin. Dette kan føre til at noen av artiklene i Cristin ikke kom med i dette prosjektet pga. manglende tilgang.

\subsection{Ekstrahering av z-verdier}
Siden p-verdier ofte er på formen $p < 0.05$ er det vanskelig å lage et histogram ut fra disse. Derfor brukes kun statistiske tester der man har test-statistikken tilgjengelig. Deretter konverteres den til en z-verdi med tilsvarende p-verdi. Selve ekstraheringen av de statistiske testene ble gjort ved hjelp av verktøyet JATSdecoder og funksjonen \texttt{get.stats()} \parencite{jatsdecoder}. Skal sies at z-curve også har denne funksjonaliteten, men ved egen testing og i følge \textcite{jatsdecoder} sine forsøk så vil JATSdecoder hente ut en større andel av de statistiske testene i artiklene.

\subsection{Modellering av replikasjonsrate og visualisering}
For å finne forventet replikasjonrate og for å lage grafer for visualisering ble de statistiske testene som ble hentet ut ved hjelp av JATSdecoder puttet inn i z-curve. Den både regnet ut forventet replikasjonrate og laget graf over de tilsvarene z-verdiene for de ulike statistikkene i artiklene. For mer informasjon om hvordan z-curve fungerer, se \textcite{z-curve-modell} og \textcite{z-curve-implementasjon}. 
 
\printbibliography

\end{document}